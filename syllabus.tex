\documentclass[11pt,letterpaper]{article}

\usepackage[margin=1in]{geometry}
\usepackage{termcal}
\usepackage{enumitem}
\usepackage[colorlinks=true, allcolors=blue]{hyperref}
\usepackage{color}
\usepackage{multirow}
\usepackage{multicol}

\newcommand{\todo}[1]{\textcolor{red}{TODO: #1}}

\title{16-745: Optimal Control and Reinforcement Learning}
\author{Spring 2021}
\date{}

\begin{document}

\maketitle

\section*{Course Description}

This is a course about how to make robots move through and interact with their environment with speed, efficiency, and robustness. We will survey a broad range of topics from nonlinear dynamics, linear systems theory, classical optimal control, numerical optimization, state estimation, system identification, and reinforcement learning. The goal is to provide students with hands-on experience applying each of these ideas to a variety of robotic systems so that they can use them in their own research.

\medskip
\noindent
\textbf{Prerequisites:} Strong linear algebra skills, experience with a high-level programming language like Python, MATLAB, or Julia, and basic familiarity with ordinary differential equations.

\section*{Instructors}

\begin{center}
\begin{tabular}{l l}
	Prof. Zac Manchester & \textbf{Email:} \href{mailto:zacm@cmu.edu}{zacm@cmu.edu} \\
	TA: Brian Jackson & \textbf{Email:} \href{mailto:bjackso2@andrew.cmu.edu}{bjackso2@andrew.cmu.edu}
	\\
	TA: Tanmay Shankar & \textbf{Email:} \href{mailto:tshankar@andrew.cmu.edu}{tshankar@andrew.cmu.edu}
\end{tabular}
\end{center}

\section*{Logistics}

\begin{itemize}
	\item Lectures will be held Tuesdays and Thursdays 6:00--7:20 PM Eastern time on Zoom.
	\item Office hours will be \todo{based on survey}.
	\item Homework assignments will be due by 11:59 PM Eastern time on Wednesdays. Two weeks will be given to complete each assignment.
	\item GitHub will be used to distribute and collect assignments.
	\item Slack will be used for general discussion and Q\&A outside of class and office hours.
	\item There will be no exams. Instead, each student will complete a project on a topic of their choice.
\end{itemize}

\section*{Learning Objectives}
By the end of this course, students should be able to do the following:
\begin{enumerate}
	\item Analyze the stability of dynamical systems
	\item Design LQR controllers that stabilize equilibria and trajectories
	\item Use offline trajectory optimization to design trajectories for nonlinear systems
	\item Use online convex optimization to implement model-predictive control
	\item Understand the effects of stochasticity and model uncertainty
	\item Directly optimize feedback policies when good models are unavailable
	
\end{enumerate}

\section*{Learning Resources}

There is no textbook required for this course. Video recordings of lectures and lecture notes will be posted online. Additional references for further reading will be provided with each lecture.

\section*{Homework}

Homework will be posted every 2 weeks and students will be given at least one full week to complete assignments. All homework will be distributed and collected using GitHub. Solutions and grades will be returned within one week of homework due dates.

\section*{Grading}

Grading will be based on:
\begin{itemize}
	\item 60\% Project
	\item 30\% Homeworks
	\item 10\% Participation
\end{itemize}
Attendance during lectures is not required to earn a full participation grade. Students can also participate through any combination of office hours, Slack discussions, project presentations, and by offering constructive feedback about the course to the instructors.


\section*{Course Policies}

\textbf{Late Homework:} Students are allowed a budget of 2 late days for turning in homework with no penalty throughout the semester. They may be used together on one assignment, or separately on two assignments. Beyond these two days, no other late homework will be accepted.

\medskip
\noindent
\textbf{Accommodations for Students with Disabilities:} If you have a disability and are registered with the Office of Disability Resources, I encourage you to use their online system to notify me of your accommodations and discuss your needs with me as early in the semester as possible. I will work with you to ensure that accommodations are provided as appropriate. If you suspect that you may have a disability and would benefit from accommodations but are not yet registered with the Office of Disability Resources, I encourage you to contact them at \href{mailto:access@andrew.cmu.edu}{access@andrew.cmu.edu}.

\medskip
\noindent
\textbf{Statement of Support for Students' Health \& Well-Being:} Take care of yourself. Do your best to maintain a healthy lifestyle this semester by eating well, exercising, avoiding drugs and alcohol, getting enough sleep, and taking some time to relax. This will help you achieve your goals and cope with stress.

\medskip
\noindent
If you or anyone you know experiences any academic stress, difficult life events, or feelings like anxiety or depression, we strongly encourage you to seek support. Counseling and Psychological Services (CaPS) is here to help: call 412-268-2922 and visit \href{http://www.cmu.edu/counseling}{http://www.cmu.edu/counseling}. Consider reaching out to a friend, faculty, or family member you trust for help getting connected to the support that can help.

\medskip
\noindent
\textit{If you or someone you know is feeling suicidal or in danger of self-harm, call someone immediately, day or night:}

\textit{CaPS: 412-268-2922}

\textit{Re:solve Crisis Network: 888-796-8226}

\medskip
\noindent
\textit{If the situation is life threatening, call the police:}

\textit{On campus: CMU Police: 412-268-2323}

\textit{Off campus: 911}

\medskip
\noindent
If you have questions about this or your coursework, please let me know. Thank you, and have a great semester.


\section*{Tentative Schedule}

\begin{tabular}{c|c|c|c}
	Week & Dates & Topics & Assignments \\
	\hline
	\multirow{2}{*}{1} & Feb 2 & Course Overview, Dynamics Intro & Survey \\
	 & Feb 4 & Discrete-Time Dynamics &  \\
	\hline
	\multirow{2}{*}{2} & Feb 9 & Optimization Intro & HW 1 Out \\
	 & Feb 11 & Unconstrained Numerical Optimization &  \\
	\hline
	\multirow{2}{*}{3}  & Feb 16 & Constrained Numerical Optimization &  \\
	 & Feb 18 & Linear Systems \& LQR &  \\
	\hline
	\multirow{2}{*}{4}  & Feb 23 & \textcolor{red}{No Class} & HW 1 Due \\
	 & Feb 25 & LQR as a QP and the Riccati Equation & HW 2 Out \\
	\hline
	\multirow{2}{*}{5}  & Mar 2 & Dynamic Programming & \\
	 & Mar 4 & Convex MPC &  \\
	\hline
	\multirow{2}{*}{6}  & Mar 9 & Nonlinear Trajectory Optimization \& DDP &   HW2 Due \\
	 & Mar 11 & DDP with Constraints & HW3 Out \\
	\hline
	\multirow{2}{*}{7}  & Mar 16 & Direct Collocation \& SQP  & \\
	 & Mar 18 & Floating Base Systems \& Attitude & \\
	\hline
	\multirow{2}{*}{8}  & Mar 23 & Optimization with Quaternions & \\
	 & Mar 25 & Contact Intro & \\
	\hline
	\multirow{2}{*}{9}  & Mar 30 & Hybrid Trajectory Optimization for Legged Systems & HW3 Due \\
	 & Apr 1 & \textcolor{red}{No Class} &   \\
	\hline
	\multirow{2}{*}{10}  & Apr 6 & Model Uncertainty: Iterative Learning Control &  HW4 Out \\
	 & Apr 8 & Stochastic Optimal Control \& LQG & \\
	 \hline
	\multirow{2}{*}{11}  & Apr 13 & Robust Control \& Minimax DDP &  \\
	 & Apr 15 & \textcolor{red}{No Class} &   \\
	 \hline
	\multirow{2}{*}{12}  & Apr 20 & Practical Tips and Tricks \& Control History &  HW4 Due \\
	 & Apr 22 & Optimal Control in the Wild: SpaceX \& JPL &   \\
	 \hline
	\multirow{2}{*}{13}  & Apr 27 & Optimal Control in the Wild: Autonomous Driving &  \\
	 & Apr 29 & Optimal Control in the Wild: Legged Robots &   \\
	 \hline
	\multirow{2}{*}{14}  & May 4 & Project Presentations &  \\
	 & May 6 & Project Presentations &   \\
\end{tabular}

\section*{Project Guidelines}

Students should work in groups of 1--4 to complete a substantial final project. The goal is for students to apply the coarse content to their own research. Project proposals will be solicited in March and topics will be selected in consultation with the instructors.

\medskip
\noindent
Project grades will be based on a short presentation given during the last week of class and a final report submitted via \href{https://forms.gle/j1xhW13DuvcYLf6j7}{Google drive} by May 18 \href{https://time.is/Anywhere_on_Earth}{Anywhere on Earth}. Reports should be written in the form of a 6 page (plus references) ICRA or IROS conference paper using the standard \href{https://www.ieee.org/conferences/publishing/templates.html}{two-column IEEE format}. Sections should include an abstract, introduction and/or background to motivate your problem, 2--3 main technical sections on your contributions, conclusions, and references. Grading will be based on the following criteria:
\newline
\newline
\begin{tabular}{|c|l|}
\hline
10\% & Class presentation \\
\hline
10\% & Adherence to IEEE formatting and length requirements \\
\hline
10\% & Innovation \& Creativity: Is what you did new/cool/interesting? Convince me. \\
\hline
30\% & Clarity of presentation: Can I understand what you did from your writing + plots? \\
\hline
40\% & Technical correctness: Are your results reasonable? Is your code correct? \\
\hline	
\end{tabular}


\end{document}
